% 请确保文件编码为utf-8,使用XeLaTex进行编译,或者通过overleaf进行编译

\documentclass[answers]{exam}  % 使用此行带有作答模块
% \documentclass{exam} % 使用此行只显示题目

\usepackage{xeCJK}
\usepackage{zhnumber}
\usepackage{graphicx}
\usepackage{hyperref}
\usepackage{amsmath}
\usepackage{amssymb}
\usepackage{mathtools}
\usepackage{booktabs}
\usepackage{enumerate}
\usepackage{algorithm}
\usepackage{algpseudocode}
\usepackage[dvipsnames]{xcolor}  % 更全的色系
\usepackage{listings}
\lstset{
    language = Python,
    backgroundcolor = \color{blue!5},    % 背景色
    basicstyle = \small\ttfamily,           % 基本样式 + 小号字体
    rulesepcolor= \color{gray},             % 代码块边框颜色
    breaklines = true,                  % 代码过长则换行
    numbers = left,                     % 行号在左侧显示
    numberstyle = \small,               % 行号字体
    keywordstyle = \color{orange},            % 关键字颜色
    commentstyle =\color{green!100},        % 注释颜色
    stringstyle = \color{red!100},          % 字符串颜色
    frame = shadowbox,                  % 用(带影子效果)方框框住代码块
    showspaces = false,                 % 不显示空格
    columns = fixed,                    % 字间距固定
    %escapeinside={<@}{@>}              % 特殊自定分隔符:<@可以自己加颜色@>
    morekeywords = {as},                % 自加新的关键字(必须前后都是空格)
    deletendkeywords = {compile}        % 删除内定关键字;删除错误标记的关键字用deletekeywords删!
}

\title{2023秋季高级机器学习 \\ 习题二}
\date{2023.11.03}
\pagestyle{headandfoot}
\firstpageheadrule
\firstpageheader{南京大学}{2023秋季高级机器学习}{习题二}
\runningheader{南京大学}
{2023秋季高级机器学习}
{习题二}
\runningheadrule
\firstpagefooter{}{第\thepage\ 页(共\numpages 页)}{}
\runningfooter{}{第\thepage\ 页(共\numpages 页)}{}

% no box for solutions
% \unframedsolutions

\setlength\linefillheight{.5in}

% \renewcommand{\solutiontitle}{\noindent\textbf{答:}}
\renewcommand{\solutiontitle}{\noindent\textbf{解:}\par\noindent}

\renewcommand{\thequestion}{\zhnum{question}}
\renewcommand{\questionlabel}{\thequestion .}
\renewcommand{\thepartno}{\arabic{partno}}
\renewcommand{\partlabel}{\thepartno .}

\def\dist{{\mathrm{dist}}}
\def\x{{\boldsymbol{x}}}
\def\w{{\boldsymbol{w}}}


\begin{document}
% \normalsize
\maketitle

\begin{questions}

\question [30] \textbf{稀疏学习}

上一次习题中涉及到了PCA的矩阵低秩近似角度理解。
Robust PCA在此基础上增加了一个变量和正则项:
\begin{align}
    \min_{X^{\prime},E} \quad & \operatorname{rank}(X^{\prime})+\lambda\|E\|_0 \\
    s.t.\quad & X = X^\prime + E \nonumber
\end{align}
其中$\|\cdot\|_0$为零范数。$\lambda$为正则化参数。
为了解该优化问题,我们考虑它的凸松弛(Convex Relaxation):
\begin{align}
    \min_{X^{\prime},E} \quad & \|X^\prime\|_*+\lambda\|E\|_1 \\
    s.t.\quad & X = X^\prime + E \nonumber
\end{align}
其中$\|\cdot\|_*$为核范数(Nuclear Norm)。
使用增广拉格朗日方法(Augmented Lagrangian Method)处理约束条件,可以得到:
\begin{align}
    \min_{X^{\prime},E} \quad & \|X^\prime\|_*+\lambda\|E\|_1 + \langle Y, X - X' - E\rangle + \frac{\mu}{2}\|X-X^\prime-E\|^2_F
\end{align}
其中$Y$为拉格朗日乘子。此处省略后续的推导过程和收敛性分析,求解该优化问题的交替求解算法中的python代码片段如下:
\begin{lstlisting}
...
Xk = np.zeros(self.X.shape)
Ek = np.zeros(self.X.shape)
Yk = np.zeros(self.X.shape)
while (err > _tol) and iter_ < max_iter:
    Xk = self.nuclear_prox(self.X - Ek + self.mu_inv * Yk, self.mu_inv)
    Ek = self.L1_prox(self.X - Xk + self.mu_inv * Yk, self.mu_inv * self.lmbda)
    Yk = Yk + self.mu * (self.X - Xk - Ek)
    err = self.frobenius_norm(self.X - Xk - Ek)
    iter_ += 1
    if (iter_ % iter_print) == 0 or iter_ == 1 or iter_ > max_iter or err <= _tol:
        print('iteration: {0}, error: {1}'.format(iter_, err))
...
\end{lstlisting}

\begin{parts}
    \part [4] Robust PCA增加的变量$E$和正则项对模型有什么作用?
    \part [14] 代码片段中第6行和第7行调用的方法实现了什么优化方法解了哪两个优化问题?(写出优化方法并分别写出优化问题)
    \part [12] 你认为Robust PCA具有哪些实际应用场景?在这些应用场景中有什么优势?(举出三个具体例子,并简要说明在这些场景中的优势,多于三个批改时以前三个为准)
\end{parts}



\begin{solution}
	\begin{parts}
		\part 
		\part 
		\part 
	\end{parts}
\end{solution}

\question[30] \textbf{图半监督学习}

多标记图半监督学习算法的正则化框架如下(另见西瓜书p303)。

\begin{equation}
\mathcal{Q}(F)=\frac{1}{2}\left(\sum_{i, j=1}^{n} W_{i j}\left\Vert\frac{1}{\sqrt{d_{i}}} F_{i}-\frac{1}{\sqrt{d_{j}}} F_{j}\right\Vert^{2}\right)+\mu \sum_{i=1}^{n}\left\|F_{i}-Y_{i}\right\|^{2}
\end{equation}
\begin{parts}
	\part  [15] 求正则化框架的最优解$F^*$。
	\part  [15] 试说明该正则化框架与书中p303页多分类标记传播算法之间的关系。
\end{parts}

\begin{solution}
\begin{parts}
	\part
	\part
\end{parts}
\end{solution}

\question [40] \textbf{半监督SVM实践}

参照教材中图13.4所示的TSVM算法,在所提供的半监督数据集上进行训练,报告模型在未标记数据集以及测试集上的性能。

本次实验的数据集为一个二分类的数据集,已提前划分为训练数据和测试数据,其中训练数据划分为有标记数据和无标记数据。数据的特征维度为30,每一维均为数值类型。数据文件的具体描述如下:
\begin{itemize}
    \item \texttt{label\_X.csv,label\_y.csv}分别是有标记数据的特征及其标签。
    \item \texttt{unlabel\_X.csv,unlabel\_y.csv}分别是无标记数据的特征及其标签。
    \item \texttt{test\_X.csv,test\_y.csv}分别是测试数据的特征及其标签。
\end{itemize}

注意,训练阶段只可以使用\texttt{label\_X.csv,label\_y.csv,unlabel\_X.csv}中的数据,其他的数据只可以在测试阶段使用。
\begin{parts}
    \part  本次实验要求使用Python3.8以上编写,代码统一集中在\texttt{tsvm\_main.py}中,通过运行该文件就可以完成训练和测试,并输出测试结果。(提交一个额外的python文件)
    \part 本次实验需要完成以下功能:
    \begin{parts}
        \part [15] 参照教材中图13.4,使用代码实现TSVM算法。要求:
        \begin{enumerate}[1.]
            \item 不允许直接调用相关软件包中的半监督学习方法。
            \item 可以直接调用相关软件包的SVM算法。
            \item 可以使用诸如\texttt{cvxpy}等软件包求解QP问题。
        \end{enumerate}
        \part [10] 使用训练好的模型在无标记数据和测试数据上进行预测,报告模型在这两批数据上的准确率和ROC曲线以及AUC值。
        \part [15] 尝试使用各种方法提升模型在测试集上的性能,例如数据预处理,超参数调节等。报告你所采取的措施,以及其所带来的提升。
    \end{parts}
\end{parts}

\begin{solution}

\end{solution}


\end{questions}

\end{document}